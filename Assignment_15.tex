\documentclass[journal,12pt,twocolumn]{IEEEtran}

\usepackage{setspace}
\usepackage{gensymb}

\singlespacing


\usepackage[cmex10]{amsmath}

\usepackage{amsthm}

\usepackage{mathrsfs}
\usepackage{txfonts}
\usepackage{stfloats}
\usepackage{bm}
\usepackage{cite}
\usepackage{cases}
\usepackage{subfig}

\usepackage{longtable}
\usepackage{multirow}

\usepackage{enumitem}
\usepackage{mathtools}
\usepackage{steinmetz}
\usepackage{tikz}
\usepackage{circuitikz}
\usepackage{verbatim}
\usepackage{tfrupee}
\usepackage[breaklinks=true]{hyperref}
\usepackage{graphicx}
\usepackage{tkz-euclide}
\usepackage{float}

\usetikzlibrary{calc,math}
\usepackage{listings}
    \usepackage{color}                                            %%
    \usepackage{array}                                            %%
    \usepackage{longtable}                                        %%
    \usepackage{calc}                                             %%
    \usepackage{multirow}                                         %%
    \usepackage{hhline}                                           %%
    \usepackage{ifthen}                                           %%
    \usepackage{lscape}     
\usepackage{multicol}
\usepackage{chngcntr}

\DeclareMathOperator*{\Res}{Res}

\renewcommand\thesection{\arabic{section}}
\renewcommand\thesubsection{\thesection.\arabic{subsection}}
\renewcommand\thesubsubsection{\thesubsection.\arabic{subsubsection}}

\renewcommand\thesectiondis{\arabic{section}}
\renewcommand\thesubsectiondis{\thesectiondis.\arabic{subsection}}
\renewcommand\thesubsubsectiondis{\thesubsectiondis.\arabic{subsubsection}}


\hyphenation{op-tical net-works semi-conduc-tor}
\def\inputGnumericTable{}                                 %%

\lstset{
%language=C,
frame=single, 
breaklines=true,
columns=fullflexible
}
\begin{document}


\newtheorem{theorem}{Theorem}[section]
\newtheorem{problem}{Problem}
\newtheorem{proposition}{Proposition}[section]
\newtheorem{lemma}{Lemma}[section]
\newtheorem{corollary}[theorem]{Corollary}
\newtheorem{example}{Example}[section]
\newtheorem{definition}[problem]{Definition}

\newcommand{\BEQA}{\begin{eqnarray}}
\newcommand{\EEQA}{\end{eqnarray}}
\newcommand{\define}{\stackrel{\triangle}{=}}
\bibliographystyle{IEEEtran}
\providecommand{\mbf}{\mathbf}
\providecommand{\pr}[1]{\ensuremath{\Pr\left(#1\right)}}
\providecommand{\qfunc}[1]{\ensuremath{Q\left(#1\right)}}
\providecommand{\sbrak}[1]{\ensuremath{{}\left[#1\right]}}
\providecommand{\lsbrak}[1]{\ensuremath{{}\left[#1\right.}}
\providecommand{\rsbrak}[1]{\ensuremath{{}\left.#1\right]}}
\providecommand{\brak}[1]{\ensuremath{\left(#1\right)}}
\providecommand{\lbrak}[1]{\ensuremath{\left(#1\right.}}
\providecommand{\rbrak}[1]{\ensuremath{\left.#1\right)}}
\providecommand{\cbrak}[1]{\ensuremath{\left\{#1\right\}}}
\providecommand{\lcbrak}[1]{\ensuremath{\left\{#1\right.}}
\providecommand{\rcbrak}[1]{\ensuremath{\left.#1\right\}}}
\theoremstyle{remark}
\newtheorem{rem}{Remark}
\newcommand{\sgn}{\mathop{\mathrm{sgn}}}
\providecommand{\abs}[1]{\lvert#1\vert}
\providecommand{\res}[1]{\Res\displaylimits_{#1}} 
\providecommand{\norm}[1]{\lVert#1\rVert}
%\providecommand{\norm}[1]{\lVert#1\rVert}
\providecommand{\mtx}[1]{\mathbf{#1}}
\providecommand{\mean}[1]{E[ #1 ]}
\providecommand{\fourier}{\overset{\mathcal{F}}{ \rightleftharpoons}}
%\providecommand{\hilbert}{\overset{\mathcal{H}}{ \rightleftharpoons}}
\providecommand{\system}{\overset{\mathcal{H}}{ \longleftrightarrow}}
	%\newcommand{\solution}[2]{\textbf{Solution:}{#1}}
\newcommand{\solution}{\noindent \textbf{Solution: }}
\newcommand{\cosec}{\,\text{cosec}\,}
\providecommand{\dec}[2]{\ensuremath{\overset{#1}{\underset{#2}{\gtrless}}}}
\newcommand{\myvec}[1]{\ensuremath{\begin{pmatrix}#1\end{pmatrix}}}
\newcommand{\mydet}[1]{\ensuremath{\begin{vmatrix}#1\end{vmatrix}}}
\numberwithin{equation}{subsection}
\makeatletter
\@addtoreset{figure}{problem}
\makeatother
\let\StandardTheFigure\thefigure
\let\vec\mathbf
\renewcommand{\thefigure}{\theproblem}
\def\putbox#1#2#3{\makebox[0in][l]{\makebox[#1][l]{}\raisebox{\baselineskip}[0in][0in]{\raisebox{#2}[0in][0in]{#3}}}}
     \def\rightbox#1{\makebox[0in][r]{#1}}
     \def\centbox#1{\makebox[0in]{#1}}
     \def\topbox#1{\raisebox{-\baselineskip}[0in][0in]{#1}}
     \def\midbox#1{\raisebox{-0.5\baselineskip}[0in][0in]{#1}}
\vspace{3cm}
\title{ASSIGNMENT-15}
\author{Unnati Gupta}
\maketitle
\newpage
\bigskip
\renewcommand{\thefigure}{\theenumi}
\renewcommand{\thetable}{\theenumi}
Download all python codes from 
\begin{lstlisting}
https://github.com/unnatigupta2320/Assignment_15
\end{lstlisting}
%
and latex-tikz codes from 
%
\begin{lstlisting}
https://github.com/unnatigupta2320/Assignment_15
\end{lstlisting}
%
\section{Question No. 6.7}
An electronic assembly consists of two subsystems, say, \textbf{A} and \textbf{B}. From previous testing procedures, the following probabilities are assumed to be known:\\
\pr{\textbf{A} \text{ fails}}=0.2\\
\pr{\textbf{B} \text{ fails alone}}=0.15\\
\pr{\textbf{A} \text{ and } \textbf{B} \text{ fails}}=0.15
\\
\\
Evaluate the following probabilities:
\begin{enumerate}[label=\roman*)]
    \item \pr{\textbf{A} \text{ fails}-\textbf{B} \text{ has failed}}
    \item \pr{\textbf{A} \text{ fails alone}}
\end{enumerate}

\section{Solution}
\begin{enumerate}
\item Let the events be:-
\begin{table}[ht!]
\begin{tabular}{|l|l|}
\hline
\textbf{Events} & \textbf{Description}                         \\ \hline
$E_A$ & The Event in which A fails
\\ \hline
$E_B$ & The Event in which B fails 
\\ \hline
\end{tabular}
\end{table}
\item According to question:
\begin{align}
   \pr{E_A} &=0.2
   \\
 \pr{E_A E_B}&=0.15
   \end{align}
   \item Also, it is given that:
   \begin{align}
    \pr{\textbf{B}\text{ fails alone}}&=0.15
    \\
    \therefore\pr{\textbf{B}\text{ fails alone}}&=\pr{E_B}-\pr{E_A E_B}
    \end{align}
    On comparing, we get:
    \begin{align}
    &\implies 0.15=\pr{E_B}-0.15\\
    &\implies \pr{E_B}=0.3\label{
    }
\end{align}
\begin{enumerate}[label=\roman*)]
\item 
\begin{align}
    \pr{\textbf{A} \text{ fails}-\textbf{B} \text{ has failed}}
   &=\frac{\pr{E_A E_B}}{\pr{E_B}}
 \\
  \pr{\textbf{A} \text{ fails}-\textbf{B} \text{ has failed}}&=\frac{0.15}{0.3}
   \end{align}
   \begin{align}
    \boxed{\therefore\pr{\textbf{A} \text{ fails}-\textbf{B} \text{ has failed}}=0.5}
\end{align}
\item 
\begin{align}
    \pr{\textbf{A}\text{ fails alone}}
   &=\pr{E_A}-\pr{E_A \and E_B} \\
   \pr{\textbf{A}\text{ fails alone}}&=0.2-0.15
   \end{align}
   \begin{align}
    \boxed{\therefore\pr{\textbf{A}\text{ fails alone}}=0.05}
\end{align}
\end{enumerate}
\end{enumerate}
\end{document}
